%! Author = Ian
%! Date = 2/23/2024

% Preamble
\documentclass{article}

% Packages
\usepackage{amsmath}
\usepackage{fancyhdr}
\usepackage[top=2cm, left=2cm, right=2cm, bottom=2cm]{geometry}
\usepackage{multicol}
\usepackage{amsfonts}
\setlength{\columnsep}{0.5cm}

\author{Ian Chen}
\date{\today}

% Header
\fancyhf{}
\fancyhead[L]{Ian Chen}
\fancyhead[C]{CS363M Notes}
\fancyhead[R]{ic8683}
\pagestyle{fancy}

% Document
\begin{document}
    \begin{multicols*}{2}
        \subsection*{Math Properties}
        $log_a b = \frac{log_c b}{log_c a} = \frac{1}{log_b a}$\\
        $P(A|B) = \frac{P(A \cap B)}{P(B)} = \frac{P(B|A)P(A)}{P(B)}$\\
        \subsection*{Pre-Processing}
        \subsubsection*{Data Cleaning}
        \textbf{Deletion}: Remove missing values, remove duplicates\\
        \textbf{Imputation}: Fill in missing values, mean/median/mode, similar case, forward fill,
        interpolation, default values\\
        \textbf{Outliers}: Legitimate unusual values with respect to rest of data\\
        \textbf{Noise}: Random errors and variations\\
        \subsubsection*{Feature Engineering}
        \textbf{Feature Transformation}: Transform features to make them more useful\\
        \textbf{Feature Creation}: Create new features from existing features\\
        \textbf{Normalization}: Scale features to be between 0 and 1; $\frac{x_i - x_{\min}}{x_
                {\max} - x_{\min}}$\\
        \textbf{Standardization}: Scale features to $\mu = 0$ and $\sigma = 1$; $\frac{x_i - \mu}{
            \sigma}$\\
        \textbf{Binning}: Continuous $\to$ Categorical\\
        \textbf{Encoding}: Categorical $\to$ Numerical\\
        \textbf{Sampling}: Reduce \textbf{dataset} size\\
        \textbf{Aggregation}: $(x_i, y_i, z_i) \to w_i$\\
        \textbf{Dimensionality}: Attributes/columns in single case\\
        \textbf{Curse of Dimensionality}: More dimensions $\to$ More sparse\\
        \textbf{Dimensionality Reduction}:\\
        \textit{Feature Selection}: Subset of attributes, Filter, Embedded\\
        \textit{Feature Extraction}: Decrease dimensions while keeping max variance, PCA, SVD, LDA\\
        \subsection*{Decision Trees}
        Grown recursively by partitioning data into subsets based on attribute values\\
        Forms axis-parallel hyperplanes\\
        c = number of classes, p = probability/fraction of class\\
        \textit{Entropy}: $\sum\limits_{i=1}^{c} -p_i \log_2 p_i$\\
        \textit{Gini}: $1 - \sum\limits_{i=1}^{c} p_i^2$\\
        r = parent, k = number of partitions, n$_i$ = number of records in partition$_i$, Impurity
        = (Entropy, Gini)\\
        Entropy $\to$ Information Gain, Gini $\to$ Gini Gain\\
        \textit{Gain}: $Impurity(r) - (\sum\limits_{i=1}^{k} \frac{n_i}{n} Impurity(i))$\\
        \textit{Split Info}: $-\sum\limits_{i=1}^{k} \frac{n_i}{n} \log_2 \frac{n_i}{n}$\\
        \textit{Gain Ratio}: $\frac{Info Gain}{Split Info}$\\
        \textbf{Classification}: Assign labels to objects\\
        \textit{Descriptive Modeling}: Explain, describe, summarize\\
        \textit{Predictive Modeling}: Predict label of unknown record\\
        \textbf{Split Conditions}:\\
        \textit{Continuous}: Threshold value, binning\\
        \textit{Nominal \& Ordinal}: Multiway, binary w/ grouping\\
        \textbf{Characteristics}: Inexpensive to construct, fast to test, easy to interpret,
        robust to outliers(especially when pruned), redundant/irrelevant attributes don't affect
        tree structure, eager learner\\
        \textbf{Pre-Pruning}: Stop growing tree before it's fully grown\\
        \subsection*{Linear Regression}
        Finds best fit line through data\\
        Use one-hot encoding/dummy encoding for categorical data\\
        There's also polynomial and non-linear regression\\
        \textbf{Least Squares}: Minimize SSE\\
        $\hat{\beta_0} = \bar{Y} - \hat{\beta_1}\bar{X}$\\
        $\hat{\beta_1} = \frac{\sum (X_i - \bar{X})(Y_i - \bar{Y})}{\sum (X_i - \bar{X})^2}$\\
        \textit{SSE}: $\sum_{i=1}^{n} (y_i - \hat{y_i})^2$\\
        \textit{MSE}: $\frac{SSE}{n}$\\
        \textit{RMSE}: $\sqrt{MSE}$\\
        \textit{MAE}: $\frac{\sum_{i=1}^{n} \lvert y_i - \hat{y_i} \rvert}{n}$\\
        \textit{var(mean)}: $\frac{\sum_{i=1}^{n} (y_i - \bar{y_i})^2}{n}$\\
        \textit{var(fit)}: $\frac{\sum_{i=1}^{n} (y_i - \hat{y_i})^2}{n}$\\
        \textit{$R^2$}: $\frac{var(mean) - var(fit)}{var(mean)}$\\
        \textbf{Multiple Linear Regression}: $y = \beta_0 + \beta_1 x_1 + \beta_2 x_2 + \ldots +
        \beta_d x_d$\\
        \textbf{Characteristics}: Non-decreasing(More features never make $R^2$ worse), More
        features $\neq$ better model, but can be better SSE)\\
        n = number of samples, p = number of features\\
        \textbf{Adjusted $R^2$}: $1 - \frac{(1 - R^2)(n - 1)}{n - p - 1}$\\
        \subsection*{Cross Validation}
        \textbf{Training Error}: Percentage misclassified(ex. SSE, $R^2$) on training set\\
        \textbf{Test/Generalization Error}: Percentage misclassified on test/unseen set\\
        \textbf{Holdout Method}: Split data into training and test sets; Issues- Less training
        data, overrepresentation/underrepresentation, varying performance\\
        \subsubsection*{K-Fold Cross Validation}
        Split data into k folds, build model on k-1 folds, test on 1 fold, repeat k times, used
        \textbf{ONLY} to evaluate process\\
        Error = $\frac{1}{k} \sum_{i=1}^{k} Error_i$\\
        Build final model using \textbf{ALL} data\\
        \textbf{Cross Validation}: Preprocess training set and apply exact same preprocessing to
        test set\\
        \textbf{Overfitting}: Low training error but high test error, bad generalizations\\
        \textbf{Hyperparameter Tuning}: Use validation set by partitioning training set\\
        \textit{Nested Cross Validation}: Use cross validation to tune hyperparameters, pick
        lowest error then test using entire training set\\
        Get best hyperparameter using hyperparameter tuning on entire dataset, then build final
        model using the chosen hyperparameter and entire dataset\\
        \subsection*{Nearest Neighbor}
        Usually Euclidean distance, can use weight factor $= \frac{1}{d^2}$\\
        \textbf{KNN}: Find k nearest neighbors, assign majority class\\
        k too small $\to$ overfit, k too large $\to$ underfit\\
        Find best k using cross validation\\
        \textbf{Characteristics}: Instance-based learning, lazy learner, no training(just retune
        k), slow testing, curse of dimensionality, feature selection critical\\
        \subsection*{Naive Bayes}
        X = test record(x$_1$, x$_2$, \ldots, x$_d$), C = class\\
        $P(C|X) \propto P(x_1|C) \cdot P(x_2|C) \cdot \ldots \cdot P(x_d|C) \cdot P(C)$\\
        \textbf{Requirements}: Na\"{i}ve $\to$ Independence, Binning\\
        \textbf{Laplace Smoothing}: Used for multinomial/categorical data, add 1 to numerator and
        v(options for the feature) to the denominator\\
        \textbf{Characteristics}: Fast, simple, scales with higher dimensions, independence
        assumption not always true\\
        \subsection*{Evaluating Classifiers}
        \textbf{Error Rate}: Fraction of incorrect predictions on testing set; $\frac{FP + FN}{n}$\\
        \textbf{Accuracy}: Fraction of correct predictions on test set; $\frac{TP + TN}{n}$\\
        \textbf{Confusion Matrix}: P = predicted, A = actual\\
        For cross validation, sum all confusion matrices\\
        \begin{tabular}{c|c|c}
            & $+_P$ & $-_P$ \\
            \hline
            $+_A$ & TP    & FN    \\
            $-_A$ & FP    & TN    \\
        \end{tabular}\\
        \textit{TPR/Sensitivity/Recall/Coverage}: $\frac{TP}{TP + FN}$, higher better\\
        \textit{TNR/Specificity}: $\frac{TN}{FP + TN}$, higher better\\
        \textit{FPR}: $\frac{FP}{FP + TN}$, lower better\\
        \textit{FNR}: $\frac{FN}{TP + FN}$, lower better\\
        \textit{Precision}: $\frac{TP}{TP + FP}$, higher better\\
        \textit{F-Measure}: $\frac{2 \times precision \times recall}{precision + recall}$, higher
        better\\
        \textbf{Fixing Class Imbalance}:\\
        \textit{Undersampling}: Remove some majority class samples\\
        \textit{Oversampling}: Duplicate some minority class samples\\
        \subsection*{Support Vector Machines}
        Hyperplane that maximizes margin between classes\\
        Binary classifier, can be extended to multi-class\\
        One-hot encoding for categorical data\\
        \textbf{Support Vectors}: Changes hyperplane if moved, points on margin boundary or on
        the wrong side for its class\\
        \textbf{Goal}: Minimize $\frac{\lvert\lvert w \rvert\rvert}{2}$ subject to $y_i(w \cdot
        x_i + b) \geq 1$\\
        \textbf{Soft Margin SVM}: Minimize $\frac{\lvert\lvert w \rvert\rvert}{2} + C(\sum_{i=1}^{N} \xi_i)$
        subject to $y_i(w \cdot x_i + b) \geq 1 - \xi_i$\\
        $C \to \infty$ = Hard Margin SVM\\
        Too little slack $\to$ overfit, too much slack $\to$ underfit\\
        Trade-off between margin width and incorrect classification\\
        \textbf{Kernel Methods}: Transform data into higher dimensions for linear seperability\\
        \textbf{Characteristics}: Global optimization, requires feature scaling, extendable to
        multi-class, no curse of dimensionality, needs cross validation(hyperparameters, kernel
        function, cost)\\
        \subsection*{Non-Linear Regression}
        \textbf{Regression Tree}: Decision tree with continuous output rather than
        categorical, recursive partitioning, axis-parallel\\
        \textbf{KNN Regression}: Continuous KNN with neighbors means as output\\
        \textbf{Support Vector Regression}: Fit hyperplane to minimize error,
        aka.
        points outside tube\\
        \textbf{Ensembles}: Bagging/Boosting, average of the base classifiers\\
    \end{multicols*}
\end{document}