%! Author = Ian Chen
%! Date = 2/20/2024

% Preamble
\documentclass[11pt]{article}

% Packages
\usepackage{amsmath}
\usepackage{hyperref}
\usepackage{calc}
\usepackage{amsthm}
\usepackage[margin=0.75in]{geometry}
\usepackage{amssymb}
\usepackage{graphicx}

\author{Ian Chen}
\title{Practice Problem 5}

\newtheoremstyle{description}
{25pt}{\topsep}{\normalfont}{}{\bfseries}{ \textemdash}{ }{}
\newtheoremstyle{break}
{\topsep}{\topsep}{\itshape}{}{\bfseries}{}{\newline}{}

\theoremstyle{description}
\newtheorem{problem}{Problem}
\theoremstyle{break}
\newtheorem*{answer}{Answer}

% Document
\begin{document}
    \maketitle

    \begin{problem}
        Create the confusion matrix for both models.
    \end{problem}
    \begin{answer}
        \begin{tabular}{c|c|c}
            Model 1   & Predicted(+) & Predicted(-) \\
            \hline
            Actual(+) & 3            & 2            \\
            \hline
            Actual(-) & 1            & 4            \\
        \end{tabular}
        \begin{tabular}{c|c|c}
            Model 2   & Predicted(+) & Predicted(-) \\
            \hline
            Actual(+) & 5            & 0            \\
            \hline
            Actual(-) & 4            & 1            \\
        \end{tabular}
    \end{answer}

    \begin{problem}
        Calculate the accuracy of each model.
        Which model is better on the basis of accuracy?
    \end{problem}
    \begin{answer}
        Accuracy = (TP + TN) / n\\
        Model 1: (3 + 4) / 10 = 0.7\\
        Model 2: (5 + 1) / 10 = 0.6\\
        Model 1 has a higher accuracy.
    \end{answer}

    \begin{problem}
        Calculate the true positive rate (TPR) of each model.
        Which model is better on the basis of TPR?
    \end{problem}
    \begin{answer}
        TPR = TP / (TP + FN)\\
        Model 1: 3 / (3 + 2) = 0.6\\
        Model 2: 5 / (5 + 0) = 1\\
        Model 2 has a higher TPR.
    \end{answer}

    \begin{problem}
        Calculate the F-measure (of the positive class only) for each model.
        Which model is better on the basis of F-measure?
    \end{problem}
    \begin{answer}
        precision = TP / (TP + FP)\\
        recall = TP / (TP + FN)\\
        F = $\frac{2 \times precision \times recall}{precision + recall}$\\
        precision$_1$ = $\frac{3}{3 + 1}$ = 0.75\\
        recall$_1$ = $\frac{3}{3 + 2}$ = 0.6\\
        F$_1$ = $\frac{2 \times 0.75 \times 0.6}{0.75 + 0.6}$ = 0.6667\\
        precision$_2$ = $\frac{5}{5 + 4}$ = 0.5556\\
        recall$_2$ = $\frac{5}{5 + 0}$ = 1\\
        F$_2$ = $\frac{2 \times 0.5556 \times 1}{0.5556 + 1}$ = 0.7143\\
        Model 2 has a higher F-measure.
    \end{answer}
\end{document}