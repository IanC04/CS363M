%! Author = Ian Chen
%! Date = 3/26/2024

% Preamble
\documentclass[11pt]{article}

% Packages
\usepackage{amsmath}
\usepackage{hyperref}
\usepackage{calc}
\usepackage{amsthm}
\usepackage[margin=0.75in]{geometry}
\usepackage{amssymb}
\usepackage{graphicx}

\author{Ian Chen}
\title{Practice Problem 10}

\newtheoremstyle{description}
{25pt}{\topsep}{\normalfont}{}{\bfseries}{ \textemdash}{ }{}
\newtheoremstyle{break}
{\topsep}{\topsep}{\itshape}{}{\bfseries}{}{\newline}{}

\theoremstyle{description}
\newtheorem{problem}{Problem}
\theoremstyle{break}
\newtheorem*{answer}{Answer}

% Document
\begin{document}
    \maketitle

    \begin{problem}
        Assuming some Eps, which would require a higher value for MinPts:\\
        Identifying A\&B as clusters, or identifying C\&D as clusters?
    \end{problem}
    \begin{answer}
        C \& D would require a higher minimum points, as for any given radius,
        A\&B have more points within it compared to C\&D.
    \end{answer}

    \begin{problem}
        Using an Eps and MinPts that identifies C and D as clusters,
        what areas would be identified as noise?
        And how many total clusters would be found?
    \end{problem}
    \begin{answer}
        \textbf{3}\\
        (A, B, E)- would be considered one cluster as all the regions meet
        the threshold for a cluster.\\
        (C)- meets threshold for cluster.\\
        (D)- meets threshold for cluster.\\
        F isn't part of a cluster, as the region isn't dense enough, so it'll be
        considered noise.
    \end{answer}

    \begin{problem}
        Using an Eps and MinPts that identifies A and B as clusters,
        what areas would be identified as noise?
        And how many total clusters would be found?
    \end{problem}
    \begin{answer}
        \textbf{2}\\
        (A)- meets threshold for cluster.\\
        (B)- meets threshold for cluster.\\
        C, D, E, F- All considered noise as the regions aren't dense enough for
        the points to be clustered.
    \end{answer}
\end{document}