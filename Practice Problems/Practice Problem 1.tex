%! Author = Ian Chen
%! Date = 2/1/2024

% Preamble
\documentclass[11pt]{article}

% Packages
\usepackage{amsmath}
\usepackage{hyperref}
\usepackage{calc}
\usepackage{amsthm}
\usepackage[margin=0.75in]{geometry}
\usepackage{amssymb}
\usepackage{graphicx}

\author{Ian Chen}
\title{Practice Problem 1}

\newtheoremstyle{description}
{25pt}{\topsep}{\normalfont}{}{\bfseries}{ \textemdash}{ }{}
\newtheoremstyle{break}
{\topsep}{\topsep}{\itshape}{}{\bfseries}{}{\newline}{}

\theoremstyle{description}
\newtheorem{problem}{Problem}
\theoremstyle{break}
\newtheorem*{answer}{Answer}

% Document
\begin{document}
    \maketitle

    \setcounter{problem}{-1}
    \begin{problem}
        What is the class label?
        What are the attributes?
    \end{problem}
    \begin{answer}
        Class label- Passed exam\newline
        Attributes- Passed all assignments, GPA, Language
    \end{answer}

    \begin{problem}
        Start by calculating the impurity of the parent.
        Use entropy as the measure of impurity.
    \end{problem}
    \begin{answer}
        Entropy = $\sum_{i=1}^{n} -p_i \log_2 p_i$\newline
        There are 3 NOs and 4 YESs\newline
        $p_{NO} = \frac{3}{7}$\newline
        $p_{YES} = \frac{4}{7}$\newline
        Entropy = $-p_{NO} \log_{2}( p_{NO}) - p_{YES} \log_{2}(p_{YES})$\newline
        = $-\frac{3}{7} \log_{2}(\frac{3}{7}) - \frac{4}{7} \log_{2}(\frac{4}{7})$\newline
        = 0.985228136\newline
    \end{answer}

    \begin{problem}
        Next, calculate the information gain of splitting on 'Passed All Assignments'.
        The gain of splitting on 'Passed All Assignments' will be the entropy of the parent minus
        the entropy of making this split.
        (We want to know how much the impurity decreases by making this split.)
    \end{problem}
    \begin{answer}
        \begin{tabular}{c|c|c}
            & Didn't Pass All Assignments & Passed All Assignments \\
            \hline
            NO  & 1                           & 2                      \\
            YES & 2                           & 2                      \\
        \end{tabular}\newline
        Left branch: $p_{NO} = \frac{1}{3}$, $p_{YES} = \frac{2}{3}$\newline
        Entropy = $-\frac{1}{3}log_{2}(\frac{1}{3}) - \frac{2}{3}log_{2}(\frac{2}{3})$\newline
        = 0.9182958341\newline
        Right branch: $p_{NO} = \frac{2}{4}$, $p_{YES} = \frac{2}{4}$\newline
        Entropy = $-\frac{1}{2}log_{2}(\frac{1}{2}) - \frac{1}{2}log_{2}(\frac{1}{2})$\newline
        = 1\newline
        Information Gain = Entropy(parent) - ($\sum_{i=1}^{k} \frac{n_{i}}{n}Impurity(i)$)\newline
        = 0.985228136 - $(\frac{3}{7}0.9182958341 + \frac{4}{7}1)$\newline
        = 0.0202442071\newline
    \end{answer}

    \begin{problem}
        Next, calculate the information gain of splitting on 'Language'.
        The gain of splitting on 'Language' will be the entropy of the parent minus the entropy of
        making this split.
        (We want to know how much the impurity decreases by making this split.)
    \end{problem}
    \begin{answer}
        \begin{tabular}{c|c|c|c}
            & Python & Java & C++ \\
            \hline
            NO  & 2      & 1    & 0   \\
            YES & 1      & 1    & 2   \\
        \end{tabular}\newline
        Left branch: $p_{NO} = \frac{2}{3}$, $p_{YES} = \frac{1}{3}$\newline
        Entropy = $-\frac{2}{3}log_{2}(\frac{2}{3}) - \frac{1}{3}log_{2}(\frac{1}{3})$\newline
        = 0.9182958341\newline
        Middle branch: $p_{NO} = \frac{1}{2}$, $p_{YES} = \frac{1}{2}$\newline
        Entropy = $-\frac{1}{2}log_{2}(\frac{1}{2}) - \frac{1}{2}log_{2}(\frac{1}{2})$\newline
        = 1\newline
        Right branch: $p_{NO} = \frac{0}{2}$, $p_{YES} = \frac{2}{2}$\newline
        Entropy = $-\frac{0}{2}log_{2}(\frac{0}{2}) - \frac{2}{2}log_{2}(\frac{2}{2})$\newline
        = 0\newline
        Information Gain = 0.985228136 -
        $(\frac{3}{7}\cdot 0.9182958341 + \frac{2}{7}\cdot 1 + \frac{2}{7}\cdot 0)$\newline
        = 0.3059584928\newline
    \end{answer}

    \begin{problem}
        Next, calculate the information gain of splitting on 'GPA'.
        Because GPA is a continuous attribute, we need to try different candidate threshold values.
        Determine all of the candidate thresholds, then calculate the information gain for each
        of them.
    \end{problem}
    \begin{answer}
        \begin{tabular}{c|c}
            GPA & Passed Exam \\
            \hline
            2.0 & NO          \\
            2.5 & NO          \\
            3.1 & YES         \\
            3.2 & YES         \\
            3.3 & YES         \\
            3.5 & YES         \\
            3.9 & NO          \\
        \end{tabular}
        Candidates: 2.8, 3.7\newline
        Left branch(2.8): $p_{NO} = \frac{2}{2}$, $p_{YES} = \frac{0}{2}$\newline
        Entropy = $-\frac{2}{2}log_{2}(\frac{2}{2}) - \frac{0}{2}log_{2}(\frac{0}{2})$\newline
        = 0\newline
        Right branch(2.8): $p_{NO} = \frac{1}{5}$, $p_{YES} = \frac{4}{5}$\newline
        Entropy = $-\frac{1}{5}log_{2}(\frac{1}{5}) - \frac{4}{5}log_{2}(\frac{4}{5})$\newline
        = 0.7219280949\newline
        Information Gain(2.8) = 0.985228136 - $(\frac{2}{7}\cdot 0 + \frac{5}{7}\cdot 0
        .7219280949)$\newline
        = 0.4695652111\newline
        Left branch(3.7): $p_{NO} = \frac{2}{6}$, $p_{YES} = \frac{4}{6}$\newline
        Entropy = $-\frac{2}{6}log_{2}(\frac{2}{6}) - \frac{4}{6}log_{2}(\frac{4}{6})$\newline
        = 0.9182958341\newline
        Right branch(3.7): $p_{NO} = \frac{1}{1}$, $p_{YES} = \frac{0}{1}$\newline
        Entropy = $-\frac{1}{1}log_{2}(\frac{1}{1}) - \frac{0}{1}log_{2}(\frac{0}{1})$\newline
        = 0\newline
        Information Gain(3.7) = 0.985228136 - $(\frac{6}{7}\cdot 0.9182958341 + \frac{1}{7}\cdot
        0)$\newline
        = 0.1981174211\newline
    \end{answer}

    \begin{problem}
        Impurity metrics(like Gini and entropy) favor attributes with more values.
        Because 'Language' can be split 3 ways, but 'Passed All Assignments' and 'GPA' are only
        split 2 ways, we should use gain ratio to compare the attributes, rather than just gain.
        Calculate the split info for each of the three attributes.
        With that, calculate the gain ratio for each of the three attributes.
    \end{problem}
    \begin{answer}
        Split Info = $-\sum_{i=1}^{k} \frac{n_{i}}{n}log_{2}(\frac{n_{i}}{n})$\newline
        Split Info(Language) = $-\frac{3}{7}log_{2}(\frac{3}{7}) - \frac{2}{7}log_{2}(\frac{2}{7}) - \frac{2}{7}log_{2}(\frac{2}{7})$\newline
        = 1.556656707\newline
        Split Info(Passed All Assignments) = $-\frac{3}{7}log_{2}(\frac{3}{7}) - \frac{4}{7}log_{
            2}(\frac{4}{7})$\newline
        = 0.985228136\newline
        Split Info(GPA, 2.8) = $-\frac{2}{7}log_{2}(\frac{2}{7}) - \frac{5}{7}log_{2}(\frac{5}{7})$\newline
        = 0.8631205686\newline
        Split Info(GPA, 3.7) = $-\frac{6}{7}log_{2}(\frac{6}{7}) - \frac{1}{7}log_{2}(\frac{1}{7})$\newline
        = 0.5916727786\newline
        Gain Ratio = $\frac{Information Gain}{Split Info}$\newline
        Gain Ratio(Language) = $\frac{0.3059584928}{1.556656707}$\newline
        = 0.1965484692\newline
        Gain Ratio(Passed All Assignments) = $\frac{0.0202442071}{0.985228136}$\newline
        = 0.02054773545\newline
        Gain Ratio(GPA, 2.8) = $\frac{0.4695652111}{0.8631205686}$\newline
        = 0.5440320022\newline
        Gain Ratio(GPA, 3.7) = $\frac{0.1981174211}{0.5916727786}$\newline
        = 0.3348428866\newline
    \end{answer}
\end{document}