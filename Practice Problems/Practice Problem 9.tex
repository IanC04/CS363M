%! Author = Ian Chen
%! Date = 3/21/2024

% Preamble
\documentclass[11pt]{article}

% Packages
\usepackage{amsmath}
\usepackage{hyperref}
\usepackage{calc}
\usepackage{amsthm}
\usepackage[margin=0.75in]{geometry}
\usepackage{amssymb}
\usepackage{graphicx}

\author{Ian Chen}
\title{Practice Problem 9}

\newtheoremstyle{description}
{25pt}{\topsep}{\normalfont}{}{\bfseries}{ \textemdash}{ }{}
\newtheoremstyle{break}
{\topsep}{\topsep}{\itshape}{}{\bfseries}{}{\newline}{}

\theoremstyle{description}
\newtheorem{problem}{Problem}
\theoremstyle{break}
\newtheorem*{answer}{Answer}

% Document
\begin{document}
    \maketitle

    \begin{problem}
        Given the above data points, perform a k-means clustering on this dataset using Euclidean
        distance as the distance function.
        Here, k is chosen to be 3.
        The initial centroids are randomly selected as (1,1), (1,2), (3,1).
        Show the steps of the algorithm until convergence.
        What are the final clusters and their final centroids?
    \end{problem}
    \begin{answer}
        % TODO Add answer
    \end{answer}
\end{document}